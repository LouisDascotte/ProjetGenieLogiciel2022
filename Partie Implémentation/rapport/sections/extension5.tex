\documentclass[../rapport.tex]{subfiles}

\begin{document}

\subsubsection{Introduction}

L'extension d'auto-production d'énergie permet de surveiller la production d'électricité et de gérer les certificats verts.

\vspace{4mm}

Pour surveiller la production d'électricité, l'utilisateur peut ajouter un point de production d'électricité qui est associé a un compteur numérique. Pour fonctionner le point de production d'électricité doit être accepter par le fournisseur. Le point de production d'énergie fonctionne en toute transparence avec les fonctionnalités des points de fournitures de l'application de base.

\vspace{4mm}

Pour les certificats verts, l'utilisateur peut en demander lorsque le seuil de production d'électricité à depassser les 1000 kWh, dès lors une requête est envoyé au fournisseur et peut accepter le certificat vert ou non.
L'application permet de voir tous les certificats verts demandés et acquis.

\vspace{4mm}

Malheuresement, par manque de temps, la partie sur l'analyse de rendement énergetique n'as pas pu être faite.

\subsubsection{API}

La base de données de l'application de base a dû légèrement être modifier pour pouvoir garder l'historique des certificats verts.
Pour les points de productions, la base de données n'a pas dû être modifier.

Dans l'api, les méthodes des points de fournitures ont pu être utiliser pour les points de production d'énergie.

\subsubsection{Application client}

Pour la partie front-end de l'application, le graphique de la partie de base à été réutiliser pour pouvoir afficher la production d'énergie. Comme pour les points de fournitures, on peut aussi voir la production d'électricité en tableau.
Un tableau a été créer pour afficher les informations des certificats verts.

\end{document}