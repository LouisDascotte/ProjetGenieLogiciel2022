\documentclass[../rapport.tex]{subfiles}

\begin{document}

Les applications client et fournisseur sont basées sur le même modèle de données. Les entités principales sont les Client et Employee qui représentent respectivement un utilisateur client et un utilisateur employé d'un fournisseur. Les employées sont des utilisateurs particuliers car représentent les employés des différents fournisseurs, qui sont représentés par l'entité Supplier. Elle permet d'obtenir le nom et la région d'opération de chaque fournisseur. Chaque utilisateur est représenté par un identifiant unique. Lors de la création de leur compte, l'entité ClientAccount est créée pour le client et EmployeeAccount pour l'employé. ClientAccount permet d'interagir avec les Portfolios, les Meters (les compteurs) et de régler les différents paramètres pour l'utilisateur.  EmployeeAccount fonctionne de la même manière, avec des attributs différents tels que les listes de Contracts (contrats) conclus entre les clients et le fournisseur. Les Portfolio des Clients permettent de stocker des points de fournitures en lien avec des propriétés. Cela permettra à l'utilisateur de surveiller sa consommation pour un point de fourniture donnée représenté par l'attribut EAN. Chaque Contract est lié à un Meter (compteur), stockant les Readings (relevés), permettant ainsi d'avoir la consommation du compteur, et donc du point de fourniture, grâce à l'EAN qui est unique et à la date du relevé. Chaque employé possède un historique d'assignation de compteurs. On utilise une entité Notification pour les notifications, que ce soit vers le fournisseur ou le client.

\end{document}
